\section{Analisi Esplorativa dell'Harware}

La configurazione hardware adottata per la sperimentazione si avvale di due sensori distinti: una camera Mapir Survey 3 e una camera di profondità Intel RealSense D435i.

\subsection{Mapir Survey 3}

\begin{figure}[htbp]
	\centering
	\includegraphics[width=\textwidth]{ch-analisi-esplorativa-harware/images/mapir.png}
	\caption{Mapir Survey 3 (sinistra); Target di calibrazione per riflettanza (centro); Sensore luminosità (destra)}
	\label{fig:mapir-hardware}
\end{figure}

La Mapir Survey 3, visibile in fig. \ref{fig:mapir-hardware}, consiste in una camera multispettrale compatta ed economicamente accessibile, progettata per acquisire dati oltre lo spettro del visibile. La sua versatilità ne permette l'utilizzo sia su piattaforme aeree, come i droni, sia in prossimità del terreno su rover o trattori agricoli. La versione in dotazione ad AgriSky è equipaggiata con un filtro RGN, configurazione che abilita la cattura di immagini nello spettro del rosso, del verde e del vicino infrarosso. Nello specifico, le bande spettrali acquisite sono centrate sulle lunghezze d'onda del Red (660 nm), Green (550 nm) e NIR (850 nm). Il cuore del sistema è un sensore CMOS Sony Exmor R IMX117, capace di acquisire immagini a 12 MP con una risoluzione di 4000x3000 pixel.

A supporto della camera multispettrale, il setup comprende due dispositivi accessori per la radiometria. Il primo è un target di calibrazione (T4-R50) contenente quattro pannelli a riflettanza nota. Questi pannelli sono realizzati con materiali che approssimano superfici ideali a riflettanza lambertiana, ovvero superfici che riflettono la luce uniformemente in tutte le direzioni. Il loro utilizzo è cruciale in fase di post-elaborazione per convertire i valori digitali grezzi in valori di riflettanza reale, indipendentemente dalle condizioni di luce al momento dello scatto. Il secondo dispositivo è un sensore di luce ambientale (DAQ-A-SD) che assiste ulteriormente la fase di post-elaborazione misurando l'irradianza solare incidente in tempo reale, permettendo così di compensare le variazioni di luminosità.

A partire da queste tre sorgenti di dati (immagini della camera, target di calibrazione e letture del sensore di luce ambientale), il software proprietario Mapir Camera Control (MCC) elabora le informazioni per generare foto radiometricamente corrette. Poiché le specifiche operazioni matematiche di correzione non sono note essendo il codice \emph{closed source}, in tutte le analisi presentate in questo report sono state utilizzate le immagini già calibrate tramite il software ufficiale Mapir, considerate come \emph{ground truth}.

\subsection{Intel RealSense D435i}

La Intel RealSense D435i è una depth-stereo camera ampiamente impiegata in contesti di robotica industriale, automazione e per compiti di navigazione. Questo dispositivo è progettato per calcolare la distanza degli oggetti dalla camera stessa (\emph{depth}) e per ottenere questa misura in diverse condizioni di illuminazione, la camera utilizza una tecnologia di visione stereo attiva, che richiede l'impiego di sensori capaci di operare nello spettro dell'infrarosso per leggere pattern proiettati invisibili all'occhio umano.

Dal punto di vista ottico e spettrale, l'architettura della D435i è composta da tre elementi principali:

\begin{itemize}
	\item Un modulo RGB basato sul sensore OmniVision OV2740, che acquisisce lo spettro visibile (400--700 nm) ed è dotato di un filtro IR-Cut per bloccare l'infrarosso e garantire la fedeltà cromatica.
	\item Un modulo stereo composto da due sensori OmniVision OV9282 con otturatore globale (Global Shutter). Questi sensori sono privi di filtro IR-Cut e possiedono una banda passante che copre sia il visibile (400--700 nm) sia il vicino infrarosso, con un picco di trasmissione specifico tra 840 e 860 nm.
	\item Un proiettore a infrarossi basato su tecnologia laser VCSEL, che emette un pattern statico alla lunghezza d'onda di 850 nm ($\pm$ 10 nm) con una potenza media di 360 mW. Questo proiettore serve ad aggiungere texture alle superfici povere di dettagli per facilitare il calcolo della profondità.
\end{itemize}
